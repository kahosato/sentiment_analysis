%- Sentiment analysis
%	- why it is interesting
%- this report documents an experiment to replicate the work presented in \cite{pang2002thumbs}.
%	- one of the first papers to present a statistical approach to sentiment analysis
%	- serves as a baseline
%	- The dataset was provided on the framework of the module L90 in MPhil Advanced Computer Science at University of Cambridge.
%- Two baseline systems were proposed for this task. 
%	- symbolic
%	- a simple machine learning technique, in particular naive bayes
%- terminology ??
The Internet is flooded with a huge amount of personal opinion on various kinds, from a blog post on the recent U.S. election to a short review for a purchased item on e-commerce website. Understanding what are in the mind of the population may bring a significant monetary or political value, and this gave rise to an active research in {\em sentiment analysis}, application of natural language processing which aims to mine subjective information. \\
This report documents an experiment to replicate the work presented in \cite{wilson2005recognizing} and \cite{pang2002thumbs}. The former takes two symbolic approaches, using a lexicon where each word is associated with a sentiment it represents. The latter was one of the first papers to present a statistical approach to sentiment analysis. Following the description provided in these papers, I build classifiers that take a movie review and judges whether it expresses a positive or negative opinion. The lexicon and the dataset was provided on the framework of the module L90 in MPhil Advanced Computer Science at University of Cambridge. These systems will serve as baselines to the further parts of the task, where I will build my own classifier.\\
The report proceeds as following: %%TODO